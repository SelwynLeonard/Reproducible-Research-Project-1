\documentclass[]{article}
\usepackage{lmodern}
\usepackage{amssymb,amsmath}
\usepackage{ifxetex,ifluatex}
\usepackage{fixltx2e} % provides \textsubscript
\ifnum 0\ifxetex 1\fi\ifluatex 1\fi=0 % if pdftex
  \usepackage[T1]{fontenc}
  \usepackage[utf8]{inputenc}
\else % if luatex or xelatex
  \ifxetex
    \usepackage{mathspec}
  \else
    \usepackage{fontspec}
  \fi
  \defaultfontfeatures{Ligatures=TeX,Scale=MatchLowercase}
\fi
% use upquote if available, for straight quotes in verbatim environments
\IfFileExists{upquote.sty}{\usepackage{upquote}}{}
% use microtype if available
\IfFileExists{microtype.sty}{%
\usepackage{microtype}
\UseMicrotypeSet[protrusion]{basicmath} % disable protrusion for tt fonts
}{}
\usepackage[margin=1in]{geometry}
\usepackage{hyperref}
\hypersetup{unicode=true,
            pdftitle={Coursera Reproducible Research Project 1},
            pdfauthor={Selwyn Leonard},
            pdfborder={0 0 0},
            breaklinks=true}
\urlstyle{same}  % don't use monospace font for urls
\usepackage{color}
\usepackage{fancyvrb}
\newcommand{\VerbBar}{|}
\newcommand{\VERB}{\Verb[commandchars=\\\{\}]}
\DefineVerbatimEnvironment{Highlighting}{Verbatim}{commandchars=\\\{\}}
% Add ',fontsize=\small' for more characters per line
\usepackage{framed}
\definecolor{shadecolor}{RGB}{248,248,248}
\newenvironment{Shaded}{\begin{snugshade}}{\end{snugshade}}
\newcommand{\KeywordTok}[1]{\textcolor[rgb]{0.13,0.29,0.53}{\textbf{#1}}}
\newcommand{\DataTypeTok}[1]{\textcolor[rgb]{0.13,0.29,0.53}{#1}}
\newcommand{\DecValTok}[1]{\textcolor[rgb]{0.00,0.00,0.81}{#1}}
\newcommand{\BaseNTok}[1]{\textcolor[rgb]{0.00,0.00,0.81}{#1}}
\newcommand{\FloatTok}[1]{\textcolor[rgb]{0.00,0.00,0.81}{#1}}
\newcommand{\ConstantTok}[1]{\textcolor[rgb]{0.00,0.00,0.00}{#1}}
\newcommand{\CharTok}[1]{\textcolor[rgb]{0.31,0.60,0.02}{#1}}
\newcommand{\SpecialCharTok}[1]{\textcolor[rgb]{0.00,0.00,0.00}{#1}}
\newcommand{\StringTok}[1]{\textcolor[rgb]{0.31,0.60,0.02}{#1}}
\newcommand{\VerbatimStringTok}[1]{\textcolor[rgb]{0.31,0.60,0.02}{#1}}
\newcommand{\SpecialStringTok}[1]{\textcolor[rgb]{0.31,0.60,0.02}{#1}}
\newcommand{\ImportTok}[1]{#1}
\newcommand{\CommentTok}[1]{\textcolor[rgb]{0.56,0.35,0.01}{\textit{#1}}}
\newcommand{\DocumentationTok}[1]{\textcolor[rgb]{0.56,0.35,0.01}{\textbf{\textit{#1}}}}
\newcommand{\AnnotationTok}[1]{\textcolor[rgb]{0.56,0.35,0.01}{\textbf{\textit{#1}}}}
\newcommand{\CommentVarTok}[1]{\textcolor[rgb]{0.56,0.35,0.01}{\textbf{\textit{#1}}}}
\newcommand{\OtherTok}[1]{\textcolor[rgb]{0.56,0.35,0.01}{#1}}
\newcommand{\FunctionTok}[1]{\textcolor[rgb]{0.00,0.00,0.00}{#1}}
\newcommand{\VariableTok}[1]{\textcolor[rgb]{0.00,0.00,0.00}{#1}}
\newcommand{\ControlFlowTok}[1]{\textcolor[rgb]{0.13,0.29,0.53}{\textbf{#1}}}
\newcommand{\OperatorTok}[1]{\textcolor[rgb]{0.81,0.36,0.00}{\textbf{#1}}}
\newcommand{\BuiltInTok}[1]{#1}
\newcommand{\ExtensionTok}[1]{#1}
\newcommand{\PreprocessorTok}[1]{\textcolor[rgb]{0.56,0.35,0.01}{\textit{#1}}}
\newcommand{\AttributeTok}[1]{\textcolor[rgb]{0.77,0.63,0.00}{#1}}
\newcommand{\RegionMarkerTok}[1]{#1}
\newcommand{\InformationTok}[1]{\textcolor[rgb]{0.56,0.35,0.01}{\textbf{\textit{#1}}}}
\newcommand{\WarningTok}[1]{\textcolor[rgb]{0.56,0.35,0.01}{\textbf{\textit{#1}}}}
\newcommand{\AlertTok}[1]{\textcolor[rgb]{0.94,0.16,0.16}{#1}}
\newcommand{\ErrorTok}[1]{\textcolor[rgb]{0.64,0.00,0.00}{\textbf{#1}}}
\newcommand{\NormalTok}[1]{#1}
\usepackage{graphicx,grffile}
\makeatletter
\def\maxwidth{\ifdim\Gin@nat@width>\linewidth\linewidth\else\Gin@nat@width\fi}
\def\maxheight{\ifdim\Gin@nat@height>\textheight\textheight\else\Gin@nat@height\fi}
\makeatother
% Scale images if necessary, so that they will not overflow the page
% margins by default, and it is still possible to overwrite the defaults
% using explicit options in \includegraphics[width, height, ...]{}
\setkeys{Gin}{width=\maxwidth,height=\maxheight,keepaspectratio}
\IfFileExists{parskip.sty}{%
\usepackage{parskip}
}{% else
\setlength{\parindent}{0pt}
\setlength{\parskip}{6pt plus 2pt minus 1pt}
}
\setlength{\emergencystretch}{3em}  % prevent overfull lines
\providecommand{\tightlist}{%
  \setlength{\itemsep}{0pt}\setlength{\parskip}{0pt}}
\setcounter{secnumdepth}{0}
% Redefines (sub)paragraphs to behave more like sections
\ifx\paragraph\undefined\else
\let\oldparagraph\paragraph
\renewcommand{\paragraph}[1]{\oldparagraph{#1}\mbox{}}
\fi
\ifx\subparagraph\undefined\else
\let\oldsubparagraph\subparagraph
\renewcommand{\subparagraph}[1]{\oldsubparagraph{#1}\mbox{}}
\fi

%%% Use protect on footnotes to avoid problems with footnotes in titles
\let\rmarkdownfootnote\footnote%
\def\footnote{\protect\rmarkdownfootnote}

%%% Change title format to be more compact
\usepackage{titling}

% Create subtitle command for use in maketitle
\newcommand{\subtitle}[1]{
  \posttitle{
    \begin{center}\large#1\end{center}
    }
}

\setlength{\droptitle}{-2em}
  \title{Coursera Reproducible Research Project 1}
  \pretitle{\vspace{\droptitle}\centering\huge}
  \posttitle{\par}
  \author{Selwyn Leonard}
  \preauthor{\centering\large\emph}
  \postauthor{\par}
  \predate{\centering\large\emph}
  \postdate{\par}
  \date{23 August 2018}


\begin{document}
\maketitle

Loading the data

\begin{Shaded}
\begin{Highlighting}[]
\KeywordTok{library}\NormalTok{(ggplot2)}
\KeywordTok{library}\NormalTok{(plyr)}

\NormalTok{##reading/loading the data into R}
\NormalTok{activity <-}\StringTok{ }\KeywordTok{read.csv}\NormalTok{(}\StringTok{"activity.csv"}\NormalTok{,}\DataTypeTok{header=}\OtherTok{TRUE}\NormalTok{,}\DataTypeTok{stringsAsFactors =} \OtherTok{FALSE}\NormalTok{,}\DataTypeTok{strip.white =} 
           \OtherTok{TRUE}\NormalTok{,}\DataTypeTok{sep =} \StringTok{','}\NormalTok{)}
\end{Highlighting}
\end{Shaded}

Processing the activity data

\begin{Shaded}
\begin{Highlighting}[]
\NormalTok{activity}\OperatorTok{$}\NormalTok{day <-}\StringTok{ }\KeywordTok{weekdays}\NormalTok{(}\KeywordTok{as.Date}\NormalTok{(activity}\OperatorTok{$}\NormalTok{date))}
\NormalTok{activity}\OperatorTok{$}\NormalTok{DateTime <-}\StringTok{ }\KeywordTok{as.POSIXct}\NormalTok{(activity}\OperatorTok{$}\NormalTok{date, }\DataTypeTok{format =} \StringTok{"%Y%m%d"}\NormalTok{)}

\NormalTok{##pulling the data without nas}
\NormalTok{clean <-}\StringTok{ }\NormalTok{activity[}\OperatorTok{!}\KeywordTok{is.na}\NormalTok{(activity}\OperatorTok{$}\NormalTok{steps),]}
\end{Highlighting}
\end{Shaded}

\subsection{What is the mean total number of steps taken per
day?}\label{what-is-the-mean-total-number-of-steps-taken-per-day}

Calculate the total number of steps taken per day

\begin{Shaded}
\begin{Highlighting}[]
\NormalTok{sumTable <-}\StringTok{ }\KeywordTok{aggregate}\NormalTok{(activity}\OperatorTok{$}\NormalTok{steps }\OperatorTok{~}\StringTok{ }\NormalTok{activity}\OperatorTok{$}\NormalTok{date, }\DataTypeTok{FUN=}\NormalTok{sum)}
\KeywordTok{colnames}\NormalTok{(sumTable)<-}\StringTok{ }\KeywordTok{c}\NormalTok{(}\StringTok{"Date"}\NormalTok{, }\StringTok{"Steps"}\NormalTok{)}
\end{Highlighting}
\end{Shaded}

Make a histogram of the total number of steps taken each day

\begin{Shaded}
\begin{Highlighting}[]
\NormalTok{## Creating a histogram of the total steps per day}
\KeywordTok{hist}\NormalTok{(sumTable}\OperatorTok{$}\NormalTok{Steps, }\DataTypeTok{breaks =} \DecValTok{5}\NormalTok{, }\DataTypeTok{xlab =} \StringTok{"Steps"}\NormalTok{, }\DataTypeTok{main =} \StringTok{"Total Steps per Day"}\NormalTok{)}
\end{Highlighting}
\end{Shaded}

\includegraphics{PA1_template_files/figure-latex/unnamed-chunk-4-1.pdf}

Calculate and report the mean and median of the total number of steps
taken per day

\begin{Shaded}
\begin{Highlighting}[]
\NormalTok{## Mean number of Steps}
\KeywordTok{as.integer}\NormalTok{(}\KeywordTok{mean}\NormalTok{(sumTable}\OperatorTok{$}\NormalTok{Steps))}
\end{Highlighting}
\end{Shaded}

\begin{verbatim}
## [1] 10766
\end{verbatim}

\begin{Shaded}
\begin{Highlighting}[]
\NormalTok{## Meadian number of Step}
\KeywordTok{as.integer}\NormalTok{(}\KeywordTok{median}\NormalTok{(sumTable}\OperatorTok{$}\NormalTok{Steps))}
\end{Highlighting}
\end{Shaded}

\begin{verbatim}
## [1] 10765
\end{verbatim}

The mean number of steps taken each day was 10 766 steps. The median
nuber of steps taken each day was 10 765 steps.

\subsection{What is the average daily activity
pattern}\label{what-is-the-average-daily-activity-pattern}

Make a time series plot (i.e.~type = ``l'') of the 5-minut interval
(x-axis) and the average number of steps taken, average across all days
(y-axis)

\begin{Shaded}
\begin{Highlighting}[]
\KeywordTok{library}\NormalTok{(plyr)}
\KeywordTok{library}\NormalTok{(ggplot2)}
\NormalTok{##pulling data without nas}
\NormalTok{clean <-}\StringTok{ }\NormalTok{activity[}\OperatorTok{!}\KeywordTok{is.na}\NormalTok{(activity}\OperatorTok{$}\NormalTok{steps),]}

\NormalTok{##create average number of steps per interval}
\NormalTok{intervalTable <-}\StringTok{ }\KeywordTok{ddply}\NormalTok{(clean, .(interval), summarize, }\DataTypeTok{Avg =} \KeywordTok{mean}\NormalTok{(steps))}

\NormalTok{##Create line plot of average number of steps per interval}
\NormalTok{p <-}\StringTok{ }\KeywordTok{ggplot}\NormalTok{(intervalTable, }\KeywordTok{aes}\NormalTok{(}\DataTypeTok{x=}\NormalTok{interval, }\DataTypeTok{y=}\NormalTok{Avg), }\DataTypeTok{xlab =} \StringTok{"Interval"}\NormalTok{, }\DataTypeTok{ylab=}\StringTok{"Average Number of Steps"}\NormalTok{)}
\NormalTok{p }\OperatorTok{+}\StringTok{ }\KeywordTok{geom_line}\NormalTok{()}\OperatorTok{+}\KeywordTok{xlab}\NormalTok{(}\StringTok{"Interval"}\NormalTok{)}\OperatorTok{+}\KeywordTok{ylab}\NormalTok{(}\StringTok{"Average Number of Steps"}\NormalTok{)}\OperatorTok{+}\KeywordTok{ggtitle}\NormalTok{(}\StringTok{"Average Number of Steps per Interval"}\NormalTok{)}
\end{Highlighting}
\end{Shaded}

\includegraphics{PA1_template_files/figure-latex/unnamed-chunk-7-1.pdf}

Which 5-minute interval, on average across all teh days in the dataset,
contains the maximum number of steps?

\begin{Shaded}
\begin{Highlighting}[]
\NormalTok{## Maximum steps by interval}
\NormalTok{maxSteps <-}\StringTok{ }\KeywordTok{max}\NormalTok{(intervalTable}\OperatorTok{$}\NormalTok{Avg)}

\NormalTok{##Which interval contains the maximum average number of steps}
\NormalTok{intervalTable[intervalTable}\OperatorTok{$}\NormalTok{Avg}\OperatorTok{==}\NormalTok{maxSteps,}\DecValTok{1}\NormalTok{]}
\end{Highlighting}
\end{Shaded}

\begin{verbatim}
## [1] 835
\end{verbatim}

The maximum number of steps for a 5-minute interval was 206 steps. The
5-minute interval which had the maximum number of steps was the 835
interval.

\subsection{Imputing missing values}\label{imputing-missing-values}

Calculate and report the total number of missing values in the dataset
(i.e.~the total number of rows with NAs)

\begin{Shaded}
\begin{Highlighting}[]
\NormalTok{## Number of NAs in the orginal data set}
\KeywordTok{nrow}\NormalTok{(activity[}\KeywordTok{is.na}\NormalTok{(activity}\OperatorTok{$}\NormalTok{steps),])}
\end{Highlighting}
\end{Shaded}

\begin{verbatim}
## [1] 2304
\end{verbatim}

The total number of rows with steps = `NA' is 2304.

Devise a strategy for filling in all of the missing values in the
dataset. The strategy does not need to be sophisticated. For example,
you could use the mean/median for that day, or the mean for the 5-minute
interval, etc.

My strategy for filling in NAs will be to subsitute the missing steps
with the average 5-minute nterval based on the day of the week.

\begin{Shaded}
\begin{Highlighting}[]
\NormalTok{## Create the average number of steps per week and interval}
\NormalTok{avgTable <-}\StringTok{ }\KeywordTok{ddply}\NormalTok{(clean, .(interval, day), summarize, }\DataTypeTok{Avg =} \KeywordTok{mean}\NormalTok{(steps))}

\NormalTok{## Create dataset with all NAs for substitution}
\NormalTok{nadata<-}\StringTok{ }\NormalTok{activity[}\KeywordTok{is.na}\NormalTok{(activity}\OperatorTok{$}\NormalTok{steps),]}
\NormalTok{## Merge NA data with average weekday interval for substitution}
\NormalTok{newdata<-}\KeywordTok{merge}\NormalTok{(nadata, avgTable, }\DataTypeTok{by=}\KeywordTok{c}\NormalTok{(}\StringTok{"interval"}\NormalTok{, }\StringTok{"day"}\NormalTok{))}
\end{Highlighting}
\end{Shaded}

\subsection{Create a new dataset that is equal to the original dataset
but with the missing data filled
in.}\label{create-a-new-dataset-that-is-equal-to-the-original-dataset-but-with-the-missing-data-filled-in.}

\begin{Shaded}
\begin{Highlighting}[]
\NormalTok{## Reorder the new substituded data in the smae format as clean data set}
\NormalTok{newdata2 <-}\StringTok{ }\NormalTok{newdata[,}\KeywordTok{c}\NormalTok{(}\DecValTok{6}\NormalTok{,}\DecValTok{4}\NormalTok{,}\DecValTok{1}\NormalTok{,}\DecValTok{2}\NormalTok{,}\DecValTok{5}\NormalTok{)]}
\KeywordTok{colnames}\NormalTok{(newdata2) <-}\StringTok{ }\KeywordTok{c}\NormalTok{(}\StringTok{"steps"}\NormalTok{, }\StringTok{"date"}\NormalTok{, }\StringTok{"interval"}\NormalTok{, }\StringTok{"day"}\NormalTok{, }\StringTok{"DateTime"}\NormalTok{)}

\NormalTok{## Merge he NA averages and non NA data together}
\NormalTok{mergeData <-}\StringTok{ }\KeywordTok{rbind}\NormalTok{(clean, newdata2)}
\end{Highlighting}
\end{Shaded}

Make a histogram of the total number of steps taken each day and
Calculate and report the mean and median total number of steps taken per
day. Do these values differ from the estimates from the first part of
the assignment? What is the impact missing data on the estimates of the
total daily number of steps?

\begin{Shaded}
\begin{Highlighting}[]
\NormalTok{## Create sum of steps per date to compare with step 1}
\NormalTok{sumTable2 <-}\StringTok{ }\KeywordTok{aggregate}\NormalTok{(mergeData}\OperatorTok{$}\NormalTok{steps }\OperatorTok{~}\StringTok{ }\NormalTok{mergeData}\OperatorTok{$}\NormalTok{date, }\DataTypeTok{FUN=}\NormalTok{sum, )}
\KeywordTok{colnames}\NormalTok{(sumTable2)<-}\StringTok{ }\KeywordTok{c}\NormalTok{(}\StringTok{"Date"}\NormalTok{, }\StringTok{"Steps"}\NormalTok{)}

\NormalTok{## Mean of Steps with NA data taken care of}
\KeywordTok{as.integer}\NormalTok{(}\KeywordTok{mean}\NormalTok{(sumTable2}\OperatorTok{$}\NormalTok{Steps))}
\end{Highlighting}
\end{Shaded}

\begin{verbatim}
## [1] 10821
\end{verbatim}

\begin{Shaded}
\begin{Highlighting}[]
\NormalTok{## Median of Steps with NA data taken care of}
\KeywordTok{as.integer}\NormalTok{(}\KeywordTok{median}\NormalTok{(sumTable2}\OperatorTok{$}\NormalTok{Steps))}
\end{Highlighting}
\end{Shaded}

\begin{verbatim}
## [1] 11015
\end{verbatim}

\begin{Shaded}
\begin{Highlighting}[]
\NormalTok{## Creating the histogram of total steps per day, categorized by data set to show impact}
\KeywordTok{hist}\NormalTok{(sumTable2}\OperatorTok{$}\NormalTok{Steps, }\DataTypeTok{breakes=}\DecValTok{5}\NormalTok{, }\DataTypeTok{xlab =} \StringTok{"Steps"}\NormalTok{, }\DataTypeTok{main =} \StringTok{"Total Steps per Day with NAs Fixed"}\NormalTok{, }\DataTypeTok{col =} \StringTok{"Black"}\NormalTok{)}
\end{Highlighting}
\end{Shaded}

\begin{verbatim}
## Warning in plot.window(xlim, ylim, "", ...): "breakes" is not a graphical
## parameter
\end{verbatim}

\begin{verbatim}
## Warning in title(main = main, sub = sub, xlab = xlab, ylab = ylab, ...):
## "breakes" is not a graphical parameter
\end{verbatim}

\begin{verbatim}
## Warning in axis(1, ...): "breakes" is not a graphical parameter
\end{verbatim}

\begin{verbatim}
## Warning in axis(2, ...): "breakes" is not a graphical parameter
\end{verbatim}

\begin{Shaded}
\begin{Highlighting}[]
\KeywordTok{hist}\NormalTok{(sumTable}\OperatorTok{$}\NormalTok{Steps, }\DataTypeTok{breaks =} \DecValTok{5}\NormalTok{, }\DataTypeTok{xlab =} \StringTok{"Steps"}\NormalTok{, }\DataTypeTok{main =} \StringTok{"Total Steps per Day with NAs Fixed"}\NormalTok{, }\DataTypeTok{col =} \StringTok{"Grey"}\NormalTok{, }\DataTypeTok{add=}\NormalTok{T)}
\KeywordTok{legend}\NormalTok{(}\StringTok{"topright"}\NormalTok{, }\KeywordTok{c}\NormalTok{(}\StringTok{"Imputed Data"}\NormalTok{, }\StringTok{"Non-Na Data"}\NormalTok{), }\DataTypeTok{fill =} \KeywordTok{c}\NormalTok{(}\StringTok{"black"}\NormalTok{, }\StringTok{"grey"}\NormalTok{))}
\end{Highlighting}
\end{Shaded}

\includegraphics{PA1_template_files/figure-latex/unnamed-chunk-14-1.pdf}

The new mean of the imputed data is 10 821 steps to the old mean of 10
766 steps. That creates a difference of 55 steps on average per day. The
new median of the imputed data is 11 015 steps compared to the old
median of the 10 765 steps. That creates a difference of 250 steps for
the median. However, the overall shape of the distribution has not
changed.

\subsection{Are there difference in activity patterns between weekdays
and
weekends?}\label{are-there-difference-in-activity-patterns-between-weekdays-and-weekends}

Create a new factor variable in the dataset with two levels -
``weekday'' and ``weekend'' including whether a given date is a weekday
or weekend day.

\begin{Shaded}
\begin{Highlighting}[]
\NormalTok{## Create new category based on the days of the week}
\NormalTok{mergeData}\OperatorTok{$}\NormalTok{DayCategory <-}\StringTok{ }\KeywordTok{ifelse}\NormalTok{(mergeData}\OperatorTok{$}\NormalTok{day }\OperatorTok\StringTok{ }\KeywordTok{c}\NormalTok{(}\StringTok{"Saturday"}\NormalTok{, }\StringTok{"Sunday"}\NormalTok{), }\StringTok{"Weekend"}\NormalTok{, }\StringTok{"Weekday"}\NormalTok{)}
\end{Highlighting}
\end{Shaded}

Make a panel plot containing a time series plot (i.e.~type = ``l'') of
the 5-minute interval (x-axis) and the average number of steps taken,
averaged across all weekday days and weekend days (y-axis).

\begin{Shaded}
\begin{Highlighting}[]
\KeywordTok{library}\NormalTok{(lattice)}
\end{Highlighting}
\end{Shaded}

\begin{Shaded}
\begin{Highlighting}[]
\NormalTok{## Summarize data by interval and type of day}
\NormalTok{intervalTable2 <-}\StringTok{ }\KeywordTok{ddply}\NormalTok{(mergeData, .(interval, DayCategory), summarize, }\DataTypeTok{Avg =} \KeywordTok{mean}\NormalTok{(steps))}

\NormalTok{##Plot data in a panel plot}
\KeywordTok{xyplot}\NormalTok{(Avg}\OperatorTok{~}\NormalTok{interval}\OperatorTok{|}\NormalTok{DayCategory, }\DataTypeTok{data=}\NormalTok{intervalTable2, }\DataTypeTok{type=}\StringTok{"l"}\NormalTok{,  }\DataTypeTok{layout =} \KeywordTok{c}\NormalTok{(}\DecValTok{1}\NormalTok{,}\DecValTok{2}\NormalTok{),}
       \DataTypeTok{main=}\StringTok{"Average Steps per Interval Based on Type of Day"}\NormalTok{, }
       \DataTypeTok{ylab=}\StringTok{"Average Number of Steps"}\NormalTok{, }\DataTypeTok{xlab=}\StringTok{"Interval"}\NormalTok{)}
\end{Highlighting}
\end{Shaded}

\includegraphics{PA1_template_files/figure-latex/unnamed-chunk-17-1.pdf}

Yes, the step activity trends are different based on whether the day
occurs on a weekend or not. This may be due to people having an
increased opportunity for activity beyond normal work hours for those
who work during the week.


\end{document}
